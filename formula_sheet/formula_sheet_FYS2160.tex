


\documentclass[a4paper, norsk, 8pt]{article}
\usepackage[utf8]{inputenc}
\usepackage[T1]{fontenc}
\usepackage{babel, textcomp, color, amssymb, tikz, subfig, float,esint}
\usepackage{amsfonts}
\usepackage{graphicx}
\usepackage{multicol}
\usepackage{bm}
\usepackage{gensymb}
\usepackage{amsmath}
\DeclareMathOperator{\arcsinh}{arcsinh}
\DeclareMathOperator{\arctanh}{arctanh}


\usepackage{physics}
\usepackage{tikz}
\usepackage{pgfplots}

\newcommand{\addPLOT}[4]{
\addplot [domain=#1:#2,samples=200,color=#3,]{#4};}
\newcommand{\addCOORDS}[1]{\addplot coordinates {#1};}
\newcommand{\addDRAW}[1]{\draw #1;}
\newcommand{\addNODE}[2]{ \node at (#1) {#2};}

%		\PLOTS{x}{y}{left}{
%			\ADDPLOT{x^2}{-2}{2}{blue}
%			\ADDCOORDS{(0,1)(1,1)(1,2)}
%		}




\definecolor{svar}{RGB}{0,0,0}
\definecolor{opgavetekst}{RGB}{109,109,109}
\definecolor{blygraa}{RGB}{44,52,59}

\hoffset = -60pt
\voffset = -95pt
\oddsidemargin = 0pt
\topmargin = 0pt
\textheight = 0.97\paperheight
\textwidth = 0.97\paperwidth

\begin{document}
\tiny
\begin{multicols*}{3}
\subsubsection*{\scriptsize Units and Values}
Avogradro's number: $N_A=6.023\times 10^{23}$ \\
Boltzmann's cosntant: $k=1.381\times 10^{-23}$ J/K\\
Boltzmann's cosntant: $k = 8.617 \times 10^{-5}$ eV/K\\
\\ $R = 8.31J/K\text{mol}$, $nR=Nk$.\\
Calorie: $1\text{cal} = 4.2$J (amount of heat needed to raise the temperature of a gram of water by $1\degree C$.)\\
Absolute zero: $0$K $=-273.15\degree C$.\\
Plank's constant: $6.626\times 10^{-34}$J/s, $4.136\times 10^{-15}$eV/s\\
One atmosphere pressure: $1$atm $=101.325$kPa.\\
One atomic mass unit: $1.6605402\times 10^{-27}$kg

\subsubsection*{\scriptsize Definitions}
\textbf{Isotherm:} constant temperature \\
\textbf{Isobar:} constant pressure \\
\textbf{Isothermal compression:} so slow that temperature does not rise \\
\textbf{Adiabatic:} so fast heat does not escape ($Q=0$) \\
\textbf{Quasi static:} Process with constant entropy ($\Delta S = 0$)\\
\textbf{Isentropic:} Adiabatic ($Q=0$) and quasistatic ($\Delta S = 0$)
\textbf{fundamental assumption of statistical mechanics:} In an isolated system in thermal equilibrium, all accessible microstates are equally probable\\
\textbf{thermodynamic limit:} Let number of particles go to infinity

\subsubsection*{\scriptsize The laws of thermodynamics}
\begin{enumerate}
  \setcounter{enumi}{0}
  \item Defines temperature
  \item Defines energy $(U)$, and energy transformation $(W, Q)$.
  \item Defines entropy $S$
  \item Fixes the value of entropy at $T=0$K.
\end{enumerate}


\subsubsection*{\scriptsize Important formulas}
\begin{align*}
  \Omega_{total}(N, n) = \Omega_A(N, n)\Omega_B(N, n)\\
  %P = T \left( \pdv{S}{V} \right)_{U, N},
  %\frac{1}{T} = \left( \pdv{S}{U} \right)_{N, V} \quad (\text{isolated system}) \\
  %\text{Clausius} \quad \Delta S \geq \frac{\mbox{d} Q}{T} \rightarrow T\mbox{d} S = \mbox{d}U + p\mbox{d}V \quad \text{reversible}\\
  \Delta S = 0 \quad \rightarrow \quad \text{reversible} \\
  \Delta S \geq 0 \quad \rightarrow \quad \text{irrreversible} \\
  %\mbox{d}U = T\mbox{d}S - P\mbox{d}V \\
  %\mbox{d}U = T\mbox{d}S - P\mbox{d}V  + \mu \mbox{d}N\\
  %\pdv{F}{N} = \mu \\
  %C_P = \left(\pdv{H}{T}\right)_P
\end{align*}

\subsubsection*{\scriptsize Important facts}
\begin{itemize}
  \item Macrostates with maximum multiplicity are the most likely and they correspond to the average values
  \item $\Delta S$ is path independent, choose the simplest path between the states to compute
\end{itemize}

\subsection*{\footnotesize  Mol}
For a given substance, find it's atmoic mass, in units of $u$ (hydrogen $1$u, oxygen $16$u, lead $207$u). Then the mole is the mass of the substance in gram (for example $1$g), and divide with the atomic mass (for example $16$u). The substance is then $1/16$mole.


\subsection*{\footnotesize  Ideal Gas}
Low density gas $PV=NkT$, ignore particle interactions. $\langle v^2 \rangle = 3kT/m$. Compression $W =NkT\ln{V_i/V_f}$.
For indistinhuishable particles (we get $1/N!$)
\begin{align*}
  \Omega(U, V, N) &= f(N)V^NU^{3N/2} \approx \frac{V^N}{N!(3N/2)!}(2\pi mU/h^2)^{3N/2} \\
  S &= Nk\left[\ln\left( \frac{V}{N}\left( \frac{4\pi m U}{3Nh^2} \right)^{3/2}  \right) + \frac{5}{2}\right] \,\, (\text{Sackur-Tetrode})\\
  \Delta S &= Nk\ln{V_f/V_i} \qquad (\text{$U$, $N$ fixed})
\end{align*}


\subsection*{\footnotesize  Equipartition theorem}
For each quadratic degree of freedom (kinetic $v^2$, spring $k^2$, rotational $\omega^2$, etc.) contributes to $1/2kT$ to the average energy at equillibrium at temperature $T$. For $N$ particles with $f$ degrees of freedom $U = NfkT/2$. Safest to apply in changes of energy. Coutning $f$: one for each dimension, $2$ from rotation (as one axis is symmetric), vibrateing counts twice (frozen out at room temperature) in a solid $3$ vibrational directions $f+=6$. Equipartition theorem only works for kinetic energy in liquids. Use $f=3$ for a monatmoic gas and $f=5$ for a diatomic gas.\\
Derived by Boltzmannfactors and converting to integral
\begin{equation*}
    Z = \frac{1}{\Delta q}\sum_q e^{-\beta c q^2}\Delta q
\end{equation*}
and calculating energy from partition function
\subsection*{\footnotesize  Heat and Work}
\begin{equation*}
  \Delta U = Q + W
\end{equation*}
Assuming quasistatic compression $W = -P\delta V$. Use pressure from POV of gas, negative work means gas is doing work, positive work means the outside is doing work on the gas.
\begin{equation*}
  W = -\int_{V_i}^{V_f}P(V) \mbox{d} V \qquad \qquad \text{quasistatic}
\end{equation*}

\subsection*{\footnotesize  Heat Capacity}
\textbf{Heat capacity:} amount of heat needed to raise temperature per degree temperature increase $C=Q/\Delta T$.\\
\textbf{Specific heat capacity:} Heat capacity per unit mass $c=C/m$.\\
\begin{equation*}
  C = (\Delta U - W)/\Delta T, C_V = \left(\pdv{U}{T}\right)_V, C_P = \left(\pdv{U}{T}\right)_P + P\left( \pdv{V}{T} \right)
\end{equation*}
At the phase transition you can put heat into system without increasing temperature, $C=\infty$. Latent heat $L=Q/m$ is the heat needed to accomplish phase transition per mass, assume constant pressure 1atm, and no other work.
\begin{align*}
  C_V &= T\left( \pdv{S}{T} \right)_V \qquad \qquad   C_P = T \left( \pdv{S}{P} \right)_V\\
  C_V &\rightarrow 0 \quad \text{as} \quad T \rightarrow 0 \\
  C_V &\rightarrow \text{constant} \quad \text{as} \quad T \rightarrow \infty
\end{align*}


\subsection*{\footnotesize  Adiabatic compression}
So fast no heat flows out, still quasistatic, $\Delta U = W$.
$$V^{\gamma}P = \text{constant }, \gamma = (f+2)/f$$

\subsection*{\footnotesize  Two state system}
Multiplicity of the macrostate with $N$ particles where $N_{\uparrow}$ point up
\begin{equation*}
  \Omega(N, n) = \frac{N!}{N_{\uparrow}! (N-N_{\uparrow})!} =  \frac{N!}{N_{\uparrow}! N_{\downarrow}!}
\end{equation*}

\subsection*{\footnotesize  Einstein solid}
$N/3$ atoms as harmonic oscillators which can oscillate in $3$ dimensions with $q=(U_n-N\hbar\omega/2)/\hbar\omega$ units of energy
\begin{equation*}
  \Omega(N, q) = \begin{pmatrix} q+N-1 \\ q \end{pmatrix} = \frac{(q+N-1)!}{q! (N-1)!} \approx  \frac{(q+N)!}{q! N!} N_{\downarrow}!
\end{equation*}
If a composite system: $q=q_A+q_B$ and $N=N_a+N_b$ and $\Omega = \Omega_A\Omega_B$.
\begin{align*}
  \text{ low $T$}\qquad \Omega &\approx \left(\frac{Ne}{q} \right)^q,  \qquad q << N\\
  \text{high $T$}\qquad \Omega &\approx \left(\frac{qe}{N} \right)^N,  \qquad N << q \\
  S &= kN\left[ \ln{\left( \frac{q}{N} + 1\right)} \right],\quad q = U/\hbar\omega
\end{align*}

\subsection*{\footnotesize  thermodynamic potentials}

\begin{itemize}
    \item Internal Energy: $U$    Total internal thermal energy of the system. The change in the stored energy equal the sum of «energies in transit»  $\Delta U = Q + W$ (first law of thermodynamic). The infinitesimal change in internal energy: $\mbox{d}U = \delta Q + \delta W$. For infinitesimal reversible process: $\mbox{d}U = T\mbox{d}S - P \mbox{d}V$.
    Can also be $\mbox{d}U = T\mbox{d}S -P\mbox{d}V + \mu\mbox{d}N$.\\

    \item Enthalpy: $H=U+PV$,  $\Delta F \leq W$ (at constant $T$). Energy needed to create the system out of nothing, and make room for it. \\$\mbox{d}H = T\mbox{d}S + V\mbox{d}P +\mu \mbox{d}N$

    \item Helmholtz free energy: $F=U-TS$, (constant pressure). Total energy needed to create the system, minus the heat you get from enviorment. Thermodynamic identity: $\mbox{d}F=-S\mbox{d}T-P\mbox{d}V+\mu\mbox{d}N$. Minimized in thermal bath, with only energy exchange. $\Delta F \leq W$. \\ $\mu=\left( \pdv{F}{N} \right)_{T,V}$, $S=-\left( \pdv{F}{T} \right)_{V, N}$, $P=-\left(\pdv{F}{V}\right)_{T,N}$.

    \item  Gibbs free energy: $G=U-TS+PV = \mu N$, $\Delta G \leq W_{other}$. Create system in constant temperature and pressure. Minimized when change in volume with constant pressure. $\mbox{d}G = -S\mbox{d}T + V\mbox{d}P +\mu \mbox{d}N$. \\ $\mu=\pdv{G}{N}$, $V = \pdv{G}{P}$, $-S=\pdv{G}{T}$
    \item Chemical potential, the energy increase by adding a particle in to the system with constant temperature and pressure. Diffusive equilibrium $\rightarrow$ equal chemical potentials.\\
    $\mu=\left(\pdv{G}{N}\right)_{T,P}, \, \mu=-T\left( \pdv{S}{N} \right)_{U,V}, \,\, \mu=\left( \pdv{U}{N} \right)_{S,V}$\\
    Particles tend to flow from the system with higher $\mu$ to the system with lower $\mu$.\\
    For monatomic ideal gas $\mu = -kT \ln{\left( \frac{VZ_{int}}{NV_Q} \right)}$ must be equal
\end{itemize}
At constant $U$ and $T$, $S$ increases.\\
At constant $V$ and $T$, $F$ decreases.\\
At constant $T$ and $P$, $G$ decreases.\\

\subsection*{\footnotesize  Reversible process}
At every step of the path the system is at equilibrium.\\ For infinitesimal reversible process: $\mbox{d}U = T\mbox{d}S - P \mbox{d}V$.\\
\textbf{Clausius equality:} $\mbox{d}S = \delta Q_{rev}/T$.\\
Irreversible	heat	transfer		is	smaller than the reversible	heat
exchange at	a	given	T
$$\frac{\delta Q_{irrev}}{T}  < \frac{\delta Q_{rev}}{T} $$
\textbf{Clausius inequality:} $\mbox{d}S \geq \delta Q/T$. For an isolated system entropy tends to	increase as	the system	sponteneously finds its equilibrium state $\mbox{d}S \geq 0$.
\begin{align*}
  C_V = \left( \frac{\delta Q}{\mbox{d}T} \right)_V = \left( \pdv{U}{T} \right) \rightarrow \mbox{d}U = C_V \mbox{d}T
\end{align*}

\subsection*{\footnotesize  Isolated system at equilibrium}
Multiplicity of a macrostate $\Omega(U,V,N)$ counts all equally likely accessible microstates, if the particles are indistinguishable the total number of microstates is reduced by $N!$. Boltzmann's formula $S=k\ln{\Omega} = -k\sum_{s}P(s)\ln{P(s)}$, for equilibrium state at fixed $U$, entropy is maximized at equilibrium, $\mbox{d}S=0$.
\begin{equation*}
    \mbox{d}S = \frac{1}{T}\mbox{d}U  + \frac{P}{T}\mbox{d}V - \frac{\mu}{T}\mbox{d}N
\end{equation*}
For a composite system
\begin{align*}
  \pdv{S_A}{U_A} = \pdv{S_B}{U_B} \qquad (\text{at equillibrium}) \\
  \pdv{S_A}{V_A} = \pdv{S_B}{V_B} \qquad (\text{at equillibrium}) \\
  \pdv{S_A}{N_A} = \pdv{S_B}{N_B} \qquad (\text{at equillibrium})
\end{align*}

\subsection*{\footnotesize  Extensive and Intensive Quantities}
An extensive quantity doubles when you double a system.\\
The quantities which are unchanged when you double the system are intensive.
\begin{itemize}
    \item \textbf{Extensive:} $V$, $N$, $S$, $U$, $H$, $F$, $G$, mass
    \item \textbf{Intensive:} $T$, $P$, $\mu$, density
\end{itemize}
\begin{align*}
    \text{extensive} \times \text{intensive} &= \text{extensive} \\
     \frac{\text{extensive}}{\text{extensive}} &= \text{intensive} \\
     \text{extensive} \times \text{extensive} &= \text{neither, you did something wrong} \\
     \text{extensive} + \text{extensive} &= \text{extensive} \\
     \text{extensive} + \text{intensive} &= \text{not allowed}
\end{align*}


\subsection*{\footnotesize  Van der Waals Equation}
\begin{align*}
    \left( P + \frac{aN^2}{V^2}\right)\left(V-Nb\right) = NkT
\end{align*}
$Nb$ makes it impossible to compress to zero volume. the $a$ term accounts for short range attractive forces.
Low density gas $PV=NkT$, ignore particle interactions. $\langle v^2 \rangle = 3kT/m$. Compression $W =NkT\ln{V_i/V_f}$.

\subsection*{\footnotesize Maxwell Construction}
Makes it possible to find the pressure at the phase transition from the $PV$-diagram, create straight line which makes even areas in the $PV$-line. Gives us the point of phase transition between liquid and gas. Gas at high volume, liquid at low volume. The intersected regions are unstable.\\
The phase transition happens at constant PRESSURE! We find the pressure on the P-V diagram by the equal areas construction of a given isotherm. Phase transition from liquid to vapor at a constant Gibbs free energy $G(P, T, N)$, $\mbox{d}G_{liquid} = \mbox{d}G_{gas}$.\\
\textbf{Clausius Clapeyron relation}
$$-S_l\mbox{d}T + V_l\mbox{d}P = -S_g\mbox{d}T+V_g\mbox{d}P $$
$$ \frac{\mbox{d}P}{\mbox{d}T} = \frac{S_g-S_l}{V_g-V_l} = \frac{\Delta S}{\Delta V} = \frac{L}{T\Delta V}$$
Entropy jumps going from a liquid to a gas. Volume expansion going from a liquid to a gas. The Clausius Clapeyron relation tells us how much the phase transition pressure changes with changing temperature.

\subsection*{\footnotesize  Sterlings approximation}
\begin{align*}
  N! &\approx N^Ne^{-N}\sqrt{2\pi N} \\
  N! &\approx N^Ne^{-N} \\
  \ln{(N!)} &\approx N\ln{N}-N \\
  \ln{\binom{N}{n}}  &\approx N\ln{N}-n\ln{n}-(N-n)\ln{\left(N-n\right)}
\end{align*}


\subsection*{\footnotesize  Maxwell relations}
Functions are usually well behavied, so order of differential does not matter, can then use
\begin{equation*}
    \pdv{}{V}\left( \pdv{U}{S} \right) = \pdv{}{S}\left( \pdv{U}{V} \right)
\end{equation*}
By using thermodynamic identities for d$U$ one can derive the maxwell relations, here they are
\begin{align*}
  \pdv{U}{S}{V} \rightarrow \left( \pdv{T}{V} \right)_S &= - \left( \pdv{P}{S} \right)_V \\
  \pdv{H}{S}{P} \rightarrow \left( \pdv{T}{P} \right)_S &= + \left( \pdv{V}{S} \right)_P \\
  -\pdv{F}{T}{V} \rightarrow \left( \pdv{S}{V} \right)_T &= + \left( \pdv{P}{T} \right)_V \\
  \pdv{G}{T}{P} \rightarrow \left( \pdv{S}{P} \right)_T &= - \left( \pdv{V}{T} \right)_P \\
\end{align*}

\subsection*{\footnotesize  Maxwell distribution}
Derived from multiplying the probability of having velocity $v$, and the number of vectors corresponding to velocity $v$. Probability is $e^{-mv^2/2kT}$. Number of vectors corresponding to $v$ is $4\pi v^2$. Find $D(v)$ by normalizing, $\int_{0}^{\infty}\mbox{d}v D^{3D}(v)=1$. get
\begin{equation*}
    D(v) = \left(\frac{m}{2\pi k T}\right)^{3/2}4\pi v^2 e^{-mv^2/2kT}
\end{equation*}


\subsection*{\footnotesize  Boltzmann statistics}
\subsubsection*{\scriptsize Thermal Bath}
Can only exchange energy, system + bath is isolated. Probability for the system to be in a specific microstate at fixed $T$: $P_i = \frac{1}{Z}e^{-E_i/kT}$, with the partition function: $Z = \sum_{i} e^{-E_i/kT}$, counts all accessible microstates weighted by the Boltzmann factor. The partition function determines the thermodynamic potential which is minimized at given $T$, $V$ and $N$. Due to energy exchange with the thermal bath the equilibrium macrostate is an average
\begin{align*}
  \langle E \rangle &= -\frac{1}{Z}\pdv{Z}{\beta}  = -\pdv{}{\beta}\ln{Z}\qquad \text{equilibrium, reservoir}\\
  \langle E^2 \rangle &= -\frac{1}{Z}\pdv[2]{Z}{\beta}\quad \qquad \qquad \qquad \,\text{equilibrium, reservoir}
\end{align*}
Boltzmann distribution for the average number of particles (occupation number) in a given energy state
\begin{equation*}
  \langle N_s \rangle = NP(s) = e^{-\beta(E_s-\mu)}
\end{equation*}
We also have the following proerties
\begin{align*}
    \langle A^n \rangle &= \frac{1}{Z}\sum_{i} A^n e^{-E_i/kT} \\
    U &= N\langle E \rangle  \\
    F &= -kT\ln{Z} \\
    Z &= e^{-F/kT} \\
    S &= \frac{U-F}{T}
\end{align*}
\subsubsection*{\scriptsize Many particle partition function}
\begin{align*}
  Z_{tot} &= Z_1Z_2...Z_N\,\, \text{distinguishable, identical and independent}\\
  Z_{tot} &= \frac{Z^N}{N!}\,\, \text{indistinguishable, identical and independent }
\end{align*}
If fixed number of particles $a$ and $b$, which do not interact, the partition function
\begin{equation*}
  Z_{total} = \binom{N}{N_a}(Z_1^a)^{N_a}(Z_1^b)^{N_b}
\end{equation*}
Two systems who are independent and distinguishable
$$Z_{tot} = Z_AZ_B $$
\subsubsection*{\scriptsize Thermal Bath and particle reservoir}
System can exchange energy and particles with a reservoir. Equilibrium at a fixed $T$ and $\mu$. Probability of the system in a specific microstate at fixed $T$ and $\mu$
\begin{equation*}
  P(s) = \frac{1}{Z_g}e^{-\beta \left( E_s-\mu N_s \right)}, \qquad Z_g = \sum_s e^{-\beta \left(E_s-\mu N_s \right) }
\end{equation*}
Can be showed that the probability of the occupation number $N$ of the given state is
\begin{equation*}
  P(s) = \frac{e^{-\beta N \left( E_s-\mu  \right)}}{\sum_{N} e^{-\beta N \left( E_s-\mu  \right)}}
\end{equation*}
From this one derives the quantum statistics:
\subsection*{\footnotesize  Quantum Statistics}
\begin{align*}
    \text{Gibbs factor} &= e^{-(E_s - \mu_{s} N_{s})/kT} \\
    %\text{Gibbs factor, two particles} &= e^{-(E(i) - \mu_A(i) N_A(i))/kT} \\
    \text{Grand partition function}\,\, Z &= \sum_s e^{-(E_s - \mu_s N_s)/kT} \\
    v_Q = l_Q^3 &= \left( \frac{h}{\sqrt{2\pi mkT}} \right)^3 \\
    \text{quantum condition $Z_t=Z^N/N!$:}\,\,\,\, \frac{V}{N} &>> v_Q
\end{align*}
\textbf{\textsc{FERMIONS:}}
\begin{align*}
    Z &= 1 + e^{-(\epsilon-\mu)/kT} \\
    \langle n_{FD} \rangle &= \frac{1}{1+e^{(\epsilon-\mu)/kT}}
\end{align*}
$  \langle n_{FD} \rangle $ is the average occupation number of the given	energy state $\epsilon$ at fixed $T$ and $\mu$.
\textbf{\textsc{BOSONS:}}
\begin{align*}
    Z &= 1 + e^{-(\epsilon-\mu)/kT} + e^{-2(\epsilon-\mu)/kT} + ... \\
    \langle n_{be} \rangle &= \frac{1}{e^{(\epsilon-\mu)/kT}-1}
\end{align*}
$  \langle n_{BE} \rangle $ is the average occupation number of the given	energy state $\epsilon$ at fixed $T$ and $\mu$.
The distribution becomes the boltzmann distribution in the limit $\beta\left( \epsilon - \mu \right) \rightarrow \infty $
\subsubsection*{\scriptsize  Density of state}
Average energy:
\begin{equation*}
  U = \sum_{n_x,n_y,n_z} \langle N(\epsilon) \rangle \epsilon(n) \approx \int_0^{\infty}\mbox{d}n_{x,y,z}\epsilon \langle N \rangle = \int_{0}^{\infty}\mbox{d}\epsilon\,\, g(\epsilon)\cdot \epsilon \cdot \langle N \rangle
\end{equation*}
Average number of particles
\begin{gather*}
  N(T,V,\mu) = \int_0^{\infty} \mbox{d}\epsilon\,\, g(\epsilon) \langle N \rangle \\
  N = \int_0^{\epsilon_F} g(\epsilon) \mbox{d}\epsilon \qquad (T=0)
\end{gather*}
Density of state $g(\epsilon)$ to count all the quantum states at a given energy $\epsilon$. The number of states with energy between $\epsilon$ and $\epsilon + \mbox{d}\epsilon = $ Number of states with state number between $n$ and $n+\mbox{d}n$ (positive quadrant). Different for different dimensions
\begin{equation*}
  \boxed{g(\epsilon) \mbox{d}\epsilon=} \quad(3D) \frac{1}{8}4\pi n^2 \mbox{d}n, \quad(2D) \frac{1}{4}2\pi n \mbox{d}n, \quad(1D)\mbox{d}n
\end{equation*}
the energy is determined by quantum mechanics
\begin{itemize}
  \item particle in a box: $\epsilon(n) = h^2n^2/(8mL^2)$
  \item harmonic oscillator: $\epsilon(n) = n\hbar\omega$
  \item relativistic particles: $\epsilon(n) = hf = hcn/(2L)$
\end{itemize}
\begin{equation*}
    g(\epsilon)\mbox{d}\epsilon = D(n)\mbox{d}n \rightarrow g(\epsilon) = \frac{D(n)}{\dv{\epsilon}{n}}
\end{equation*}
\begin{gather*}
  \epsilon_F(N) = \epsilon(n_{max}) \\
  N_{3D} = 4\pi n_{max}^3/(8\cdot 3), \quad N_{2D} = 2\pi n_{max}^2/4, \quad N_{1D} = 2n_{max}\\
  U(T,V,\epsilon_F) = \int_0^{\epsilon_F} \mbox{d}\epsilon \, g(\epsilon)\, \epsilon \\
  N(T,V,\epsilon_F) = \int_0^{\epsilon_F} \mbox{d}\epsilon \, g(\epsilon) \\
\end{gather*}
For \textbf{\textsc{fermions}} multiply by $2$, spin up and spin down.\\
For \textbf{\textsc{photons}} multiply by $2$, for the two transverse polarization of EM waves, $\mu=0$.\\
For \textbf{\textsc{phonons}} multiply by $3$, for all three polarizations of the sound waves.


For a fermi gas at $T=0$ all states with $\epsilon < \epsilon_F$ are occupied, while all states with $\epsilon > \epsilon_F$ are unoccupied.
\subsection*{\footnotesize  Integrals}
$$ \int_0^{\infty} \frac{x^3}{e^{bx}-1} = \frac{\pi^4}{15b^4}$$
$$ \int_0^{\infty} \frac{x^3}{e^{bx}+1} = \frac{7\pi^4}{120b^4}$$

\subsection*{\footnotesize  Taylor expansions}
\begin{align*}
  \ln{1+x} &\approx x - x^2/2 \\
  \sinh{x} &= \sum_{n_0} \frac{x^{2n+1}}{(2n+1)!} \\
  \cosh{x} &= \sum_{n_0} \frac{x^{2n}}{2n!}\\
  \sqrt{1+x} &= 1 + \frac{x}{2} - \frac{x^2}{8}
\end{align*}

\subsection*{\footnotesize  Definitions}

\begin{align*}
  \sinh{x} &= \frac{e^{x}-e^{-x}}{2} \\
  \cosh{x} &= \frac{e^{x}+e^{-x}}{2} \\
  \cosh{\left( \arctanh{x} \right)} &= \frac{1}{\sqrt{1-x}} \\
  \cosh{\left( \arcsinh{x} \right)} &= \sqrt{1+x^2}
\end{align*}

\subsection*{\footnotesize  Tricks}
\begin{gather*}
  \sum_{n}^{N} \frac{N!}{n!(N-n)!}a^Nb^{N-n} = (a+b)^N \\
  \sum_{n}^{N} n\frac{N!}{n!(N-n)!}a^Nb^{N-n} = a\pdv{}{a}\sum_{n}^{N} \frac{N!}{n!(N-n)!}a^Nb^{N-n} \\
  Rp^{R} = p \pdv{}{p}R^{p} \\
  \frac{1}{1-x} = 1 + x + x^2 + x^3 +... \\
  \text{Surface area sphere in d dimensions} = \frac{2\pi^{d/2}}{\left( \frac{d}{2} - 1 \right)!}r^{d-1}\\
  \sum_{n=0}^{\infty} ne^{-an} = \frac{e^{a}}{\left(e^a-1\right)^2}
\end{gather*}

\subsection*{\footnotesize  Examples}
\subsubsection*{\scriptsize Degrees of freedom, H2O}
3 translational d.o.f.
Since such a molecule has no symmetry axis, like a diatomic molecule has, there are 3 rotational d.o.f.
Three possible modes/motions of vibration: two atomic bonds (connecting
the two H-atoms to the O-atom) which can stretch/contract, and there can
be a flexing motion where the angle at which the two H-atoms are bound to
the O-atom increases/decreases (the H-atoms move towards and away from
each other). For each of these three vibrational modes there are 2 d.o.f. Total $12$.

\subsubsection*{\scriptsize Quantum Ideal Gas}
\begin{align*}
  \epsilon_n &= \frac{\vec{p}\cdot\vec{p}}{2m}  = \frac{h^2}{8mL^2}\left(n_x^2+n_y^2+n_z^2\right) \\
  Z_1(T,V) &= \sum_{n_x}\sum_{n_y}\sum_{n_z}e^{\frac{-h^2\left(n_x^2+n_y^2+n_z^2\right)}{8mL^2kT}} = \left( \sum_{n}e^{\frac{-h^2n^2}{8mL^2kT}} \right)^3 \\
  Z_1(T,V) &\approx_{high T} \frac{1}{2}\int_{-\infty}^{\infty}\mbox{d}n e^{\frac{-h^2n^2}{8mL^2kT}} = \frac{V}{\Lambda^3(T)}, \Lambda = \sqrt{\frac{h}{2\pi mkT}}\\
  Z_N(T,V) &= \frac{1}{N!}\left(\frac{V}{\Lambda^3(T)}\right)^N \\
  F_N &= -kT\ln{Z_N(T,V)} = -NkT\left( \ln{\left( \frac{V}{N\Lambda^3(T)}\right)} -1 \right)\\
  \mu(T,V) &= \left( \pdv{F}{N} \right)_{T,V} = -kT\ln{\left( \frac{V}{N\Lambda^3(T)} \right)} \\
  P &= -\left( \pdv{F}{V} \right)_{T,N} = \frac{kT}{V}\\
  U &= -\pdv{}{\beta}\left( \ln{Z_N\left(T, V\right)}\right) = 3NkT/2 \\
  S &= \frac{U-F}{T} = Nk\left[ \ln{\left( \frac{V}{N\Lambda^3(T)} \right)} + \frac{5}{2} \right]
\end{align*}

\end{multicols*}
\end{document}
%kovalent binding
