


\documentclass[a4paper, norsk, 8pt]{article}
\usepackage[utf8]{inputenc}
\usepackage[T1]{fontenc}
\usepackage{babel, textcomp, color, amsmath, amssymb, tikz, subfig, float,esint}
\usepackage{amsfonts}
\usepackage{graphicx}
\usepackage{multicol}
\usepackage{bm}
\usepackage{physics}
\usepackage{tikz}
\usepackage{pgfplots}

\newcommand{\EQU}[1] { \begin{equation*} \begin{split}
#1
\end{split} \end{equation*} }
 \newcommand{\DE}[1] {  \begin{description}  #1 \end{description} }
 \newcommand{\IT}[2] { \item[\color{blue} #1]{#2} }
 \newcommand{\vv}[1] { \mathbf{#1} }
 \newcommand{\PAR}[2]{ \frac{\partial #1}{\partial #2}}
 \newcommand{\expe}[1] { \left\langle#1\right\rangle }
 \newcommand{\ket}[1] { |#1\rangle }
  \newcommand{\bra}[1] { \langle #1 | }
  \newcommand{\braket}[2] { \langle #1 | #2 \rangle }
  \newcommand{\commutator}[2]{ \left[ #1 , #2\right] }
  \newcommand{\colvec}[2] {
  \left( \begin{matrix}
 #1 \\
 #2 \\
  \end{matrix}\right) }
 \newcommand{\PLOTS}[4]{
\begin{tikzpicture}
\begin{axis}[
    axis lines = #3, %usally left
    xlabel = #1,
    ylabel = #2,
]
#4
\end{axis}
\end{tikzpicture}
}


\newcommand{\addPLOT}[4]{
\addplot [domain=#1:#2,samples=200,color=#3,]{#4};}
\newcommand{\addCOORDS}[1]{\addplot coordinates {#1};}
\newcommand{\addDRAW}[1]{\draw #1;}
\newcommand{\addNODE}[2]{ \node at (#1) {#2};}

%		\PLOTS{x}{y}{left}{
%			\ADDPLOT{x^2}{-2}{2}{blue}
%			\ADDCOORDS{(0,1)(1,1)(1,2)}
%		}




\definecolor{svar}{RGB}{0,0,0}
\definecolor{opgavetekst}{RGB}{109,109,109}
\definecolor{blygraa}{RGB}{44,52,59}

\hoffset = -60pt
\voffset = -95pt
\oddsidemargin = 0pt
\topmargin = 0pt
\textheight = 0.97\paperheight
\textwidth = 0.97\paperwidth

\begin{document}
\tiny
\begin{multicols*}{3}
\subsubsection*{\scriptsize Units and Values}
Avogradro's number: $N_A=6.023\times 10^{23}$ \\
Boltzmann's cosntant: $k=1.381\times 10^{-23}J/K$,\\ $R = 8.31J/K\text{mol}$, $nR=Nk$.\\
Calorie: $1\text{cal} = 4.2$J (amount of heat needed to raise the temperature of a gram of water by $1\degree C$.)\\
Absolute zero: $0$K $=-273.15\degree C$.


\subsubsection*{\scriptsize Definitions}
\textbf{Isotherm:} constant temperature \\
\textbf{Isobar:} constant pressure \\
\textbf{Isothermal compression:} so slow that temperature does not rise \\
\textbf{Adiabatic compression:} so fast heat does not escape \\
\textbf{fundamental assumption of statistical mechanics:} In an isolated system in thermal equilibrium, all accessible microstates are equally probable\\
\textbf{thermodynamic limit:} Let number of particles go to infinity\\


\subsubsection*{\scriptsize Important formulas}
\begin{align*}
  \Omega_{total}(N, n) = \Omega_A(N, n)\Omega_B(N, n)\\
  N! \approx N^Ne^{-N}\sqrt{2\pi N} \\
  \ln{(N!)} \approx N\ln{N}-N \\
  P = T \left( \pdv{S}{V} \right)_{U, N}
  \frac{1}{T} = \left( \pdv{S}{U} \right)_{N, V} \quad (\text{isolated system}) \\
  \text{Clausius} \quad \Delta S \geq \frac{\mbox{d} Q}{T} \rightarrow T\mbox{d} S = \mbox{d}U + p\mbox{d}V \quad \text{reversible}\\
  C_V \rightarrow 0 \quad \text{as} \quad T \rightarrow 0 \\
  C_V \rightarrow \text{constant} \quad \text{as} \quad T \rightarrow \infty \\
  \Delta S = 0 \quad \rightarrow \quad \text{reversible} \\
  \Delta S \geq 0 \quad \rightarrow \quad \text{irrreversible} \\
  \pdv{S_A}{V_A} = \pdv{S_B}{V_B} \qquad (\text{at equillibrium}) \\
  \pdv{S_A}{N_A} = \pdv{S_B}{N_B} \qquad (\text{at equillibrium}) \\
  \mbox{d}U = T\mbox{d}S - P\mbox{d}V \\
  \mbox{d}U = T\mbox{d}S - P\mbox{d}V  + \mu \mbox{d}N\\
\end{align*}

\subsubsection*{\scriptsize Important facts}
\begin{itemize}
  \item Macrostates with maximum multiplicity are the most likely and they correspond to the average values
  \item $\Delta S$ is path independent, choose the simplest path between the states to compute
\end{itemize}

\subsection*{\footnotesize  Ideal Gas}
Low density gas $PV=NkT$, ignore particle interactions. $\langle v^2 \rangle = 3kT/m$. Compression $W =NkT\ln{V_i/V_f}$.
For indistinhuishable particles (we get $1/N!$)
\begin{align*}
  \Omega(U, V, N) &= f(N)V^NU^{3N/2} \approx \frac{V^N}{N!(3N/2)!}(2\pi mU/h^2)^{3N/2} \\
  S &= Nk\left[\ln\left( \frac{V}{N}\left( \frac{4\pi m U}{3Nh^2} \right)^{3/2}  \right) + \frac{5}{2}\right] \\
  \Delta S &= Nk\ln{V_f/V_i} \qquad (\text{$U$, $N$ fixed})
\end{align*}


\subsection*{\footnotesize  Equipartition theorem}
For each quadratic degree of freedom (kinetic $v^2$, spring $k^2$, rotational $\omega^2$, etc.) contributes to $1/2kT$ to the average energy at equillibrium at temperature $T$. For $N$ particles with $f$ degrees of freedom $U = NfkT/2$. Safest to apply in changes of energy. Coutning $f$: one for each dimension, $2$ from rotation (as one axis is symmetric), vibrateing counts twice (frozen out at room temperature) in a solid $3$ vibrational directions $f+=6$. Equipartition theorem only works for kinetic energy in liquids. Use $f=3$ for a monatmoic gas and $f=5$ for a diatomic gas.

Derived by Boltzmannfactors $e^{-a\beta x^2}/Z$ and computing average of energy.

\subsection*{\footnotesize  Heat and Work}
\begin{equation}
  \Delta U = Q + W
\end{equation}
Assuming quasistatic compression $W = -P\delta V$. Use pressure from POV of gas, negative work means gas is doing work, positive work means the outside is doing work on the gas.
\begin{equation*}
  W = -\int_{V_i}^{V_f}P(V) \mbox{d} V \text{quasistatic}
\end{equation*}

\subsection*{\footnotesize  Heat Capacity}
\textbf{Heat capacity:} amount of heat needed to raise temperature per degree temperature increase $C=Q/\Delta T$.\\
\textbf{Specific heat capacity:} Heat capacity per unit mass $c=C/m$.\\
\begin{equation*}
  C = (\Delta U - W)/\Delta T, C_V = \left(\pdv{U}{T}\right)_V, C_P = \left(\pdv{U}{T}\right)_P + P\left( \pdv{V}{T} \right)
\end{equation*}
At the phase transition you can put heat into system without increasing temperature, $C=\infty$. Latent heat $L=Q/m$ is the heat needed to accomplish phase transition per mass, assume constant pressure 1atm, and no other work.



\subsection*{\footnotesize  Adiabatic compression}
So fast no heat flows out, still quasistatic, $\Delta U = W$.
V^{\gamma}P = \text{constant }, \gamma = (f+2)/f

\subsection*{\footnotesize  Two state system}
Multiplicity of the macrostate with $N$ particles where $N_{\uparrow}$ point up
\begin{equation}
  \Omega(N, n) = \frac{N!}{N_{\uparrow}! (N-N_{\uparrow})!} =  \frac{N!}{N_{\uparrow}! N_{\downarrow}!}
\end{equation}

\subsection*{\footnotesize  Einstein solid}
$N/3$ atoms as harmonic oscillators which can oscillate in $3$ dimensions with $q=(U_n-N\hbar\omega/2)/\hbar\omega$ units of energy
\begin{equation}
  \Omega(N, q) = \begin{pmatrix} q+N-1 \\ q \end{pmatrix} = \frac{(q+N-1)!}{q! (N-1)!} \approx  \frac{(q+N)!}{q! N!} N_{\downarrow}!}
\end{equation}
If a composite system: $q=q_A+q_B$ and $N=N_a+N_b$ and $\Omega = \Omega_A\Omega_B$.
\begin{align*}
  \text{ low $T$}\qquad \Omega &\approx \left(\frac{Ne}{q} \right)^q,  \qquad q << N\\
  \text{high $T$}\qquad \Omega &\approx \left(\frac{qe}{N} \right)^N,  \qquad N << q \\
  S &= kN\left[ \ln{\left( \frac{q}{N} + 1\right)} \right],\quad q = U/\hbar\omega
\end{align*}
\subsection*{\footnotesize  Van der Waals Equation}
Low density gas $PV=NkT$, ignore particle interactions. $\langle v^2 \rangle = 3kT/m$. Compression $W =NkT\ln{V_i/V_f}$.

\subsection*{\footnotesize  Enthalpy}
In addition to talking about the energy of system we add the work needed to make room for it
\begin{equation}
  H = U + PV
\end{equation}
Energy needed to create the system out of nothing, and make room for it. $\Delta H = \Delta U + P \Delta V$. $C_P = \left(\pdv{H}{T}\right)_P$

\subsection*{\footnotesize  thermodynamic potentials}
\begin{equation}
  \mu = -T\left( \pdv{S}{N} \right)
\end{equation}
\subsection*{\footnotesize  Tricks}
\begin{align*}
  \sum_{n}^{N} \frac{N!}{n!(N-n)!}a^Nb^{N-n} &= (a+b)^N \\
  \sum_{n}^{N} n\frac{N!}{n!(N-n)!}a^Nb^{N-n} &= a\pdv{}{a}\sum_{n}^{N} \frac{N!}{n!(N-n)!}a^Nb^{N-n} \\
  Rp^{R} &= p \pdv{}{p}R^{p}
\end{align*}

\subsection*{\footnotesize  Taylor expansions}
\begin{align*}
  \ln{1+x} \approx x - x^2/2
\end{align*}

\end{multicols*}
\end{document}
%kovalent binding
